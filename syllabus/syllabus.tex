\documentclass[11pt]{article}
\usepackage{fullpage}
\usepackage[left=1in,top=1in,right=1in,bottom=1in,headheight=3ex,headsep=3ex]{geometry}

\newcommand{\blankline}{\quad\pagebreak[2]}

\setlength\parindent{0pt}

\title{ECON368/DCS368: Big Data and Economics}
\author{Kyle Coombs he/him/his}
\date{Fall 2023}

\usepackage[sc]{mathpazo}
\linespread{1.05}         % Palatino needs more leading (space between lines)
\usepackage[T1]{fontenc}

\usepackage[mmddyyyy]{datetime}% http://ctan.org/pkg/datetime
\usepackage{advdate}% http://ctan.org/pkg/advdate
\newdateformat{syldate}{\twodigit{\THEMONTH}/\twodigit{\THEDAY}}
\newsavebox{\MONDAY}\savebox{\MONDAY}{Mon}% Mon

\newcommand{\week}[1]{%
%  \cleardate{mydate}% Clear date
% \newdate{mydate}{\the\day}{\the\month}{\the\year}% Store date
  \paragraph*{\kern-2ex\quad #1, \AdvanceDate[1]\syldate{\today} - \AdvanceDate[2]\syldate{\today}:}% Set heading  \quad #1
%  \setbox1=\hbox{\shortdayofweekname{\getdateday{mydate}}{\getdatemonth{mydate}}{\getdateyear{mydate}}}%
  \ifdim\wd1=\wd\MONDAY
    \AdvanceDate[7]
  \else
    \AdvanceDate[7]
  \fi%
}

\usepackage{setspace}
\usepackage{multicol}
%\usepackage{indentfirst}
\usepackage{fancyhdr,lastpage}
\usepackage{url}
\pagestyle{fancy}
\usepackage{hyperref}
%\usepackage{lastpage}
\usepackage{amsmath}
\usepackage{layout}   
\lhead{}
\chead{}
\rhead{\footnotesize Big Data And Economics  -- Fall 2023}
\lfoot{}
\cfoot{\small \thepage/\pageref*{LastPage}}
\rfoot{}

\usepackage{array, xcolor}
\usepackage{color,hyperref}
\definecolor{clemsonorange}{HTML}{EA6A20}
\hypersetup{colorlinks,breaklinks,
            linkcolor=clemsonorange,urlcolor=clemsonorange,
            anchorcolor=clemsonorange,citecolor=black}

            %latex natbib here
\usepackage{natbib}

\begin{document}

\maketitle

\begin{tabular*}{\textwidth}{@{\extracolsep{\fill}}lr}
  %%%%%%%%%%%%%%%%%%%%%%%%%%%%%%%%%%%%%%%%%%%%%%%%%%%%%%%%%%%%%%
  
  % Modify information %%%%%%%%%%%%%%%%%%%%%%%%%%%%%%%%%%%%%%%%%
E-mail: \texttt{kcoombs@bates.edu} & Web: \href{kylecoombs.com}{\tt\bf kylecoombs.com} \\

  Office Hours: T, 4-5pm, W 10:30-11:30am  (Zoom or in-person)  &  Class Hours: T/Th 11am-12:20pm \\
  Office: PGill 277 & Classroom: Bonney 310 \\ 
  \multicolumn{2}{c}{Course Website: \href{https://github.com/big-data-and-economics}{\tt\bf https://github.com/big-data-and-economics}} \\
  \multicolumn{2}{c}{OH Link: \href{https://calendar.app.google/XF36Ujpg9NcJbSD58
  }{\tt\bf https://calendar.app.google/XF36Ujpg9NcJbSD58}} \\
\hline
\end{tabular*} \\
  
\section{Deprecated Information}

Please disregard anything below. For now this is only uploaded until I can finalize changes and delete.

\subsection*{Note}
\label{sec:note}

This syllabus contains a rough outline of the course and may change in the future. If you have any questions, you should check with me. If this syllabus ever conflicts with what is listed on the course website README, the website takes precedence.

\section*{Course Description}
\label{sec:desc}

Economics is at the forefront of developing statistical methods for analyzing data collected from uncontrolled sources. Since econometrics addresses challenges in estimation such as sample selection bias and treatment effects identification, the discipline is well-suited to analyze large or unstructured datasets. This course introduces practical tools and econometric techniques used to conduct empirical analysis on topics like equality of opportunity, education, racial disparities, and more. These skills include data acquisition, project management, version control, data visualization, efficient programming, and tools for big data analysis. The course also explores how econometrics and statistical learning methods cross-fertilize and can be used to advance knowledge on topics like inequality, education, racial disparities, health care, and more where large volumes of data are rapidly accumulating. We will also cover the ethics of data collection and analysis. 

% Second Section %%%%%%%%%%%%%%%%%%%%%%%%%%%%%%%%%%%%%%%%%%%
\section*{Course Objectives}

After this course is done, you should know how to:

\begin{enumerate}
\item Organize empirical projects that are replicable, reproducible, and collaborative using good programming practices
\item Collect and clean big or novel datasets using APIs, web scraping, and other methods
\item Use Big Data to generate key insights about economic opportunity, inequality, and racial discrimination
\item Understand the differences between prediction, causality, and description, and when to apply each
\item Explain what data science is, and how Big Data differs from other types of data 
\end{enumerate}

% Third Section %%%%%%%%%%%%%%%%%%%%%%%%%%%%%%%%%%%%%%%%%%%%%s

\section*{Required Materials}
\label{sec:materials}

Course notes, assignments, extra readings, recordings, and all other materials are available on the GitHub class materials repository. \textit{The notes are adapted from Grant McDermott's course at the University of Oregon, Tyler Ransom's course at the University of Oklahoma, Raj Chetty's course at Harvard University, Nick Huntington-Klein's Econometrics course, and others listed in this syllabus.}

\subsection*{Software requirements}

All the software requirements for this course are open-source and/or free. Please aim to have \textbf{\textit{R}} and \textbf{\textit{Rstudio}} installed by the start of our first lecture. Other installation will be a part of Problem Set 0. I will be available for installation troubleshooting during the first week of the semester. If you want a detailed tutorial on how to achieve a perfect working setup, I can think of no finer guide than Jenny Bryan \textit{et al}.'s \url{http://happygitwithr.com/} (see esp. sections 4 -- 15).

\vspace{-0.25cm}
\subsubsection*{\textit{R} and RStudio}

We will mainly be using the statistical programming language \textbf{\textit{R}} (download \href{https://www.r-project.org/}{here}). 
Please make sure that you install the \textbf{RStudio IDE} too (download \href{https://www.rstudio.com/products/rstudio/download/preview/}{here}).

\vspace{-0.25cm}
\subsubsection*{Git/GitHub Desktop}

We will also make extensive use of the \textbf{Git} version control system (follow the OS-specific installation instructions \href{http://happygitwithr.com/install-git.html}{here}). Once you have installed Git, please create an account on \textbf{GitHub} (\href{https://github.com/join}{here}) and register for an education discount to get unlimited private repos (\href{https://education.github.com/discount_requests/new}{here}).\footnote{GitHub recently \href{https://blog.github.com/changelog/2019-01-08-pricing-changes/}{announced} unlimited free private repos for everyone. However, you are limited to three collaborators per private repo, so the education discount still makes sense.} Now is probably a good time to tell you that I am going to run the course through \href{https://classroom.github.com/}{GitHub Classroom}. You will receive an email invitation to the course repo with instructions in due time, but suffice it to say that this is how we'll submit assignments, provide feedback, receive grades, etc.

\vspace{-0.25cm}
\subsubsection*{LaTeX software}
\textbf{TeX Live}: \\
A LaTeX software distribution that is compatible with the LaTeX Workshop extension in Visual Studio Code. Installation instructions can be found here: \url{https://www.tug.org/texlive/}. Use the ``easy install'' option for your operating system. \\

\hspace{-0.25cm} \textbf{Overleaf}: \\
I also recommend that you create an Overleaf account. Overleaf is a useful tool for learning LaTex, which you will use to write your final. You can create an account here \url{https://www.overleaf.com/register}. Note, you can sync your Overleaf account with your GitHub account using instructions here \url{https://www.overleaf.com/learn/how-to/Git_integration}. This is a premium feature at present, so I do not require it. You can also sync it with Dropbox, which is similarly a premium feature \url{https://www.overleaf.com/learn/how-to/Dropbox_Synchronization}. (Student plans cost \$89 per year.)

\vspace{-0.25cm}
\subsubsection*{Recommended but not required:}

You are ready to start this course once you have installed R, RStudio, and Git (as well as created an account on GitHub). Make sure they are fully up-to-date.

\begin{enumerate}
  \item \textit{GitHub Copilot} by GitHub -- \url{https://marketplace.visualstudio.com/items?itemName=GitHub.copilot}
  \item \textit{ChatGPT - Genie AI} by Genie AI -- \url{https://marketplace.visualstudio.com/items?itemName=genieai.chatgpt-vscode}
  \item \textit{Anaconda} or \textit{PIP} -- largely used for Python installations, there a few quality of life packages for R that are distributed via \textit{Anaconda} or \textit{PIP}. \url{https://docs.anaconda.com/free/anaconda/install/index.html}
  \item \textit{Radian} -- Radian allows you to use RStudio similar to how you would RStudio. You will be able to run code directly into a terminal with \texttt{Ctrl+Enter}, but also have access to GitHub CoPilot coding assistance. \url{https://github.com/randy3k/radian}
\end{enumerate}

\textbf{Visual Studio Code}
VSCode is free and open-source, and is available for Windows, Mac, and Linux. You can download it at \url{https://code.visualstudio.com/download}. Once you have installed VSCode, you will need to install a variety of extensions. We will cover installations during the problem set (or as they become necessary), but here is a list:

\begin{enumerate}
  \item The \textit{R} extension by REditorSupport -- \url{https://code.visualstudio.com/docs/languages/r}
  \item \textit{LaTeX Workshop} by James Yu -- \url{https://marketplace.visualstudio.com/items?itemName=James-Yu.latex-workshop}
  \item \textit{GitHub Classroom} by GitHub -- \url{https://marketplace.visualstudio.com/items?itemName=GitHub.classroom&ssr=false#overview}
\end{enumerate}

\textbf{Operating system-specific recommendations:}

\begin{itemize}
	\item \textbf{Linux:} You should be good to go. 
	\item \textbf{Mac:} Install the \href{https://brew.sh/}{Homebrew} package manager. I also recommend that you make sure your C++ toolchain is configured/open. Don't worry, it's simpler than it sounds. Just download the \href{https://github.com/rmacoslib/r-macos-rtools#installer-package-for-macos-r-toolchain-}{macOS Rtools installer} and follow the instructions.
	\item \textbf{Windows:} Install \href{https://cran.r-project.org/bin/windows/Rtools/}{Rtools}. While its not essential, I also recommend that you install the \href{https://chocolatey.org/}{Chocolatey} package manager for Windows. Furthermore, please install the Windows Subsystem for Linux (WSL) and the Ubuntu distribution. Instructions \href{https://docs.microsoft.com/en-us/windows/wsl/install-win10}{here}.
\end{itemize}

I will provide instructions for any further software requirements as the need arises; i.e. when we get to the relevant lecture. On that note, each week's lectures will be posted by the preceding Sunday on the \href{https://github.com/ECON368-fall2023-big-data-and-economics}{course website}. Each lecture lists all the \textit{R} packages and external libraries (if relevant) required for a particular class. I'll try to remind you, but my expectation is that you will look at these requirements and ensure that you have them installed \textit{before} we start class. 

\subsection*{Textbook and other readings}

There's no set textbook for this course. I'll draw on readings from select \textit{free} sources as needed listed below. You don't \textit{need} to buy any of these (excellent) books to complete the course. But I can eagerly recommend leafing through at least one or two of them. Each of these books is freely available online if you can't afford a hard copy:
%
\subsubsection*{On R}

\begin{itemize}
    \item \href{https://r4ds.had.co.nz/}{\emph{R} For Data Science} by Hadley Wickham and Garrett Grolemund
    \item \href{https://adv-r.hadley.nz/}{Advanced \emph{R}} by Hadley Wickham
    \item \href{https://geocompr.robinlovelace.net/}{Geocomputation with \emph{R}} by Robin Lovelace, Jakub Nowosad, and Jannes Muenchow
    \item \href{https://rstudio.com/resources/cheatsheets/}{\emph{Posit} Cheatsheets}
    \item \href{https://bookdown.org/rdpeng/rprogdatascience/}{R Programming for Data Science} by Roger D. Peng
    \item \href{https://www.youtube.com/watch?v=Ih84O1vfH8Y&t=3391s}{Bates Alumni Eli Mokas and Ian Ramsay's RStudio Tutorial}
\end{itemize}

\subsubsection*{On R Markdown}

\begin{itemize}
    \item \href{https://rmarkdown.rstudio.com/gallery.html}{\emph{RStudio} Gallery}
    \item \href{https://bookdown.org/yihui/rmarkdown/}{R Markdown: The Definitive Guide} by Yihui Xie, J. J. Allaire, and Garrett Grolemund
\end{itemize}

\subsubsection*{Econometrics, Statistics, Data Science with R examples}

\begin{itemize}
    \item \href{https://statlearning.com}{An Introduction to Statistical Learning} by Gareth James, Daniela Witten, Trevor Hastie, and Robert Tibshirani
    \begin{itemize}
        \item \href{https://emilhvitfeldt.github.io/ISLR-tidymodels-labs/}{ISLR Labs}
    \end{itemize}
    \item \href{https://grantmcdermott.com/ds4e/}{Data Science for Economists and Other Animals} by Grant McDermott and Ed Rubin
    \item \href{https://mixtape.scunning.com/}{Causal Inference: The Mixtape} by Scott Cunningham
    \item \href{https://www.theeffectbook.net/}{The Effect} by Nick Huntington-Klein
    \item \href{https://r-spatial.org/book/}{Spatial Data Science} by Edzer Pebesma and Roger Bivand
    \item \href{http://socviz.co/}{Data Visualization: A practical introduction} by Kieran Healy
    \item \href{https://docs.google.com/spreadsheets/d/1yLNdpb0TkYfNN-phme1Amt4XPU1bOB6vINHam1ss_fk/edit#gid=1544370596}{Curated List by Nathan Tefft}
    \item \href{https://lost-stats.github.io/}{Library of Statistical Techniques (LOST)}
\end{itemize}

\subsubsection*{Staying organized}

\begin{itemize}
    \item \href{https://web.stanford.edu/~gentzkow/research/CodeAndData.pdf}{Code and Data for the Social Sciences: A Practitioner's Guide} by Matthew Gentzkow and Jesse Shapiro
    \item \href{https://scholar.harvard.edu/sites/scholar.harvard.edu/files/ristovska/files/coding_for_econs_20190221.pdf}{Coding for Economists: A Language-Agnostic Guide}
    \item \href{https://happygitwithr.com/}{happygitwithr} by Jenny Bryan
\end{itemize}

\subsubsection*{Large Language Models}

You are actively encouraged to use generative AI assistants in this class. These can be used to improve your code, refine your writing, iterate on your ideas, and more.

\begin{itemize}
    \item \href{https://chat.openai.com/auth/login}{Sign-up for ChatGPT}
    \item \href{https://docs.github.com/en/copilot/quickstart#signing-up-for-github-copilot-for-your-personal-account}{Sign-up for GitHub CoPilot} (Note: you do not signup through this organization, you signup through your own personal GitHub account as a student.)
    \item \href{https://raw.githack.com/tyleransom/DScourseS23/master/LectureNotes/27-GPT/27slides.html#1}{Tips to get better with ChatGPT}
    \item \href{https://intro2r.library.duke.edu/ai.html}{Integration of AI with R}
\end{itemize}

Taking a step back, one of the goals of this course (and most Data Science courses) is to make you aware of the incredible array of instruction material that is freely available online. I also want to encourage you to be entrepreneurial. In that spirit, many of the lectures will follow a tutorial on someone's blog tutorial, or involve reproducing an existing study with open source tools. Each lecture will come with a set of recommended readings, which I expect you to at least look over before class.

\section*{Prerequisites}
Prerequisites: ECON 255 and ECON 260 or ECON 270 The course assumes background in econometrics and statistics. 

\section*{Teaching Assistant}

There is no teaching assistant for this course. The course does have a Course-Attached Tutor (CAT), who is a student who has taken the course before and is available to help you with the course. The CAT for this course is \textbf{TK}. The CAT will hold office hours and review sessions. The CAT will also grade your problem sets and final project.

\section*{Student Academic Support Center}
Scheduled hours for R held in the Student Academic Support Center (SASC) of the Library are:

\begin{itemize}
  \item Sunday - 7:30-9pm
  \item Monday - 12-1pm, 2:30pm-4pm
  \item Tuesday - 12-2:30pm, 6-7:30pm
  \item Wednesday - 11am-1pm, 6-7:30pm
  \item Thursday - 12-4pm, 6-7:30pm
  \item Friday - 11am-12pm
\end{itemize}

\section*{Grading Policy}
The course will have six coding problem sets (50\%), weekly five-minute student presentations (5\%), GitHub participation (5\%), a final project (40\%), and optional participation in a Hack-A-Thon to assist the City of Lewiston, which may replace a quarter of the final project grade.

\subsection*{Improving your grade}

In an effort to incentivize you to see coding as an ongoing process of learning and improvement, I will allow you to improve the coding and presentation quality portions of your grade on any problem set. However, you cannot just copy and paste the solutions. 

Instead, you must provide carefully commented explanations of each step of the code -- whether from the solutions or of your own invention. This is a great way to learn, but it is also a lot of work. 

*Example.* You might write add a comment like this to the top of your code:

\begin{verbatim}
# Create directories, suppress warning that the directory already exists. 
suppressWarnings({
  dir.create(data)
  dir.create(documentation)
  dir.create(code)
  dir.create(output)
  dir.create(writing)
})
\end{verbatim}

\subsubsection*{Submission process}

To be eligible to resubmit to improve your grade, you must have submitted an initial version of the problem set on time.

\begin{enumerate}
\item View my feedback on the `feedback` branch of your problem set repository. 
\item Fix your problem set answers and comment your code as needed. Write "CORRECTED" in all caps next to any changes. 
\item Push changes to the `main` branch of your problem set repository.
\item Navigate to the `Issues` tab of your problem set repository and create a new issue titled "Resubmission for Problem Set X". Briefly describe your changes in the body of the issue and tag my username, @kgcsport. 
\item \textbf{Deadline for resubmissions}: All resubmissions must be pushed within one week of the solutions being posted.
\end{enumerate}
Within your own private problem set repository, you can create an `Issues` tab within the Settings tab for interfacing only with me and any group partners.

\subsubsection*{Requests for reconsideration}

On occasions, you may disagree with the grade you received on a problem set. Here are my policies for reconsideration:

\begin{itemize}
\item \textbf{Deadline for requests}: All requests for reconsideration must be submitted within one week of the solutions being posted.
\item \textbf{Full regrade}: Any request for reconsideration will result in a full regrade of your problem set. This means that your grade can go up or down.
\item \textbf{Regrading high scores}: If you scored a 90 percent or above on a problem set, I will not change your grade. This is not because I do not want to help you, but because we both have limited time and I want to focus my efforts on cases where an incorrectly graded problem set could significantly impact your grade in the course. 
\end{itemize}

If you would like reconsideration, please raise an `Issue` in your private problem set repository. Title the issue "Reconsideration request for Problem Set X". Briefly describe your request in the body of the issue and tag my username, @kgcsport. 

Within your own private problem set repository, you can create an `Issues` tab within the Settings tab for interfacing only with me and any group partners.

\medskip

Further details on assignments and grading policies are provided on the \href{https://github.com/big-data-and-economics/big-data-class-materials?tab=readme-ov-file#grading-policy}{course website}.

\newpage
\section*{Course Policies}

\subsection*{During Class}
\footnotesize{We will be doing active coding projects during class, so please bring your personal laptops. Please refrain from using computers for anything but activities related to the class. Phones are prohibited as they are rarely useful for anything in the course. Eating and drinking are allowed in class, but please refrain from it affecting the course. Try not to eat your breakfast/lunch in class as the classes are typically active.}

\subsection*{Artificial Intelligence}
\footnotesize{I encourage each of you to make use of artificial intelligence-driven digital assistants, like ChatGPT and Github CoPilot. These tools are not a substitute for your own ingenuity, but instead a complement as they are incredibly useful for tasks like coding or proofreading. Please note during assignments whether and where you used ChatGPT, as you would cite your (human) sources.}

\subsection*{Attendance Policy}
\footnotesize{For complete attendance and excused absence policies, please see \href{https://www.bates.edu/dof/course-attendance-policy-guideline-for-absences/}{\tt\bf https://www.bates.edu/dof/course-attendance-policy-guideline-for-absences/}. Attendance is expected in all lectures. Valid excuses for absence will be accepted before class. In extenuating circumstances, valid excuses with proof will be accepted after class.}

%Get policy on incompletes
\subsection*{Policies on Incomplete Grades and Late Assignments}

\footnotesize{\textbf{End of course:} If an extension is not authorized by the instructor, department, or college, an unfinished incomplete grade will automatically change to an F after either (a) the end of the next regular semester in which the student is enrolled (not including short-term), or (b) the end of 12 months if the student is not enrolled, whichever is shorter.}

\footnotesize{Incompletes that change to F will count as an attempted course on transcripts. The burden of fulfilling an incomplete grade is the responsibility of the student.}

\subsection*{Academic Integrity and Honesty}
\footnotesize{Students are required to comply with the Bates policy on academic integrity in the Code of Student Conduct at \url{https://www.bates.edu/student-conduct-community-standards/student-conduct/code-of-student-conduct/}. Don't cheat. Don't be that person. Yes, you. You know exactly what I'm talking about. See \url{https://www.bates.edu/student-conduct-community-standards/student-conduct/academic-integrity-policy/} for a detailed explanation of academic integrity.}

%Zoom policy -- maybe not allowed
\subsection*{Accommodations by Zoom}
\footnotesize{I prefer that all of you attend lecture in person, but I understand that there are sometimes unavoidable conflicts. As such, the course will have an option to tune in via Zoom for those with an excused absence related to health, family, or other unavoidable conflicts/emergencies. If you have a reason you need to attend a lecture via Zoom, please get in touch to explain the situation. If you do not get in touch and attend a lecture via Zoom without approval, I will consider it an absence. Approval can be given after the fact, but I prefer to know about hybrid attendance ahead of time. Several of you have been in touch about this option already and do not need to seek further approval.}

%TK: Accommodations
\subsection*{Accommodations for Disabilities}
\footnotesize{Reasonable accommodations will be made for students with verifiable disabilities. In order to take advantage of available accommodations, students must register with the Office of Accessible Education and Student Support (AESS) in Ladd Library G35. For more information on Bates' policy on working with students with disabilities, please see the AESS webpage on Requesting Services (\url{https://www.bates.edu/accessible-education-student-support/requesting-services/how-to-register-for-accommodations/}).}

\footnotesize{Non-Discrimination Policy Bates College provides equality of opportunity in education and employment for all students and employees. Accordingly, Bates College affirms its commitment to maintain a work environment for all employees and an academic environment for all students that is free from all forms of discrimination.}

\footnotesize{Discrimination based on race, color, religion, creed, sex, national origin, age, disability, veteran status, or sexual orientation is a violation of state and federal law and/or Bates College policy and will not be tolerated. Harassment of any person (either in the form of quid pro quo or creation of a hostile environment) based on race, color, religion, creed, sex, national origin, age, disability, veteran status, or sexual orientation also is a violation of state and federal law and/or Bates College policy and will not be tolerated. Retaliation against any person who complains about discrimination is also prohibited. Bates's policies and regulations covering discrimination, harassment, and retaliation may be accessed at \url{https://www.bates.edu/here-to-help/policies/equal-opportunity-policy/}. Any person who feels that he or she has been the subject of prohibited discrimination, harassment, or retaliation should contact the Director of Title IX \& Civil Rights Compliance and Title IX Coordinator, Gwen Lexow, at \texttt{titleix@bates.edu} or \url{https://www.bates.edu/here-to-help/make-a-report/}.}

\subsection*{Accommodations for Families}

\footnotesize{If you are a parent or guardian of a child, and you are unable to attend class and care for that child for class one day, please be in touch in case you need further accommodations. You are invited to attend the lecture via Zoom or watch it asynchronously if that will make it easier to not miss course material.}

% Course Schedule %%%%%%%%%%%%%%%%%%%%%%%%%%%%%%%%%%%%%%%%%%%

\newpage
\section*{Tentative schedule and weekly learning goals}
\label{sec:sched}
The schedule is tentative and subject to change. Each week, I will cover a specific of topic. On Tuesday, we will cover relevant data science skills. On Thursday, we will apply those skills to a specific application. Bolded readings are ``key'' readings.

\normalsize

\SetDate[01/01/2024]
\week{Week 1} Introduction to Big Data
\begin{itemize}
  \item \textbf{Skills:} Installation of \textit{R}, \textit{VSCode}, etc.
  \item \textbf{Application:} Opportunity Atlas basics
  \begin{itemize}
    \item Readings: \textbf{\cite{chetty2018opportunityatlas}}, \cite{chetty2020opportunity}, \cite{einav2014ageofbigdata}
  \end{itemize}
  \item \textit{\textbf{Problem Set 0 due Monday at Midnight}}
\end{itemize}
\week{Week 2} Coding workflow, staying organized, and version control
\begin{itemize}
  \item \textbf{Skills:} Folder structure, Git(Hub), minimally reproducible examples, Docker
  \begin{itemize}
    \item Readings: \textbf{\cite{gentzkowshapiro2014code}}, \cite{mcdermott2022docker}
    \item Watch: \url{https://www.youtube.com/watch?v=7oyiPBjLAWY} by Jenny Bryan on Refactoring
  \end{itemize}
  \item \textbf{Application:} Hidden Decisions of Researchers, Data Colada
  \begin{itemize}
    \item Readings: \textbf{\cite{huntingtonklen2021influence}}, \cite{tinyverse}, \cite{wickhamtidy}, \cite{datacolada2021groundhog}, \cite{datacolada2022groundhog}
  \end{itemize}
\end{itemize}
\week{Week 3} Gathering Data, Ethics, and Privacy
\begin{itemize}
\item \textbf{Skills:} APIs, scraping, hashing, differential privacy
\begin{itemize}
  \item Readings: \textbf{\cite{chetty2019privacy}}, \cite{abowd2019privacy}, \cite{apiintro}
\end{itemize}
\item \textbf{Application:} Nowcasting Gentrification using Yelp Data
\item \textit{\textbf{Problem Set 1 due Monday at Midnight}}
\item \textit{\textbf{Project Proposal due Monday at Midnight}}
\begin{itemize}
  \item Readings: \textbf{\cite{glaeser2018gentrification}}, \cite{glaeser2017local}
\end{itemize}
\end{itemize} 
\week{Week 4} Spatial Analysis
\begin{itemize}
  \item \textbf{Skills:} Map projections, shapefiles, \textit{sf}
  \begin{itemize}
    \item Reading: \textbf{\cite{mcdermott2023spatial}}, \cite{crs}, \cite{lovelace2019geographic}
  \end{itemize}
  \item \textbf{Application:} Neighborhoods and Mobility
  \begin{itemize}
    \item Readings: \textbf{\cite{chetty2018opportunityatlas}}
  \end{itemize}
\end{itemize}
\week{Week 5} Functions \& Parallel programming
\begin{itemize}
  \item \textbf{Skills:} Functions
  \begin{itemize}
    \item Readings: \cite{mcdermott2023functionsintro}, \cite{mcdermott2023functionsadvanced} \cite{wickham2023meta}, \cite{tidyeval}
  \end{itemize}
  \item \textbf{Skill:} Parallel Programming
  \begin{itemize}
    \item Readings: \textbf{\cite{mcdermott2023parallel}}, \cite{eddelbuettel2020parallel}, \cite{mcdermott2023parallel}
  \end{itemize}
\end{itemize}
\week{Week 6} Regression review \& Causal Inference
\begin{itemize}
  \item \textbf{Skills:} OLS, IV, Potential Outcomes
  \begin{itemize}
    \item Readings: \cite{hungtintonklein2023effect} Chapter 13, \cite{cunningham2023mixtape} Chapter 4
  \end{itemize}
  \item \textbf{Application:} Returns to Education and College Proximity
  \begin{itemize}
    \item Readings: \textbf{\cite{card1993college}}
  \end{itemize}
  \item \textit{\textbf{Problem Set 2 due Monday at Midnight}}
\end{itemize}
\week{Fall Recess} Databases on Tuesday, then rest!
\begin{itemize}
  \item \textbf{Skills:} SQL
  \item \textit{\textbf{Literature Review due Monday at midnight}}
\end{itemize}
\week{Week 7} Panel data and two-way fixed effects
\begin{itemize}
  \item \textbf{Skills:} Frisch-Waugh-Lovell Theorem, Event Studies, \textit{fixest}
  \begin{itemize}
    \item \cite{hungtintonklein2023effect} Chapters 16-18, \cite{cunningham2023mixtape} Chapters 8, 9
  \end{itemize}
  \item \textbf{Application:} Causal Effects of Neighborhoods
  \begin{itemize}
    \item Reading: \textbf{\cite{chetty2018neighborhoods}}, \textbf{\cite{chetty2019moving}}, \textbf{\cite{chetty2016moving}}
  \end{itemize}
  \item \textit{\textbf{Problem Set 3 due Monday at Midnight}}
\end{itemize}
\week{Week 8} Regression Discontinuity Design
\begin{itemize}
  \item \textbf{Skills:} RDD, McCrary Test, fuzzy RDD
  \begin{itemize}
    \item Readings: \cite{cunningham2023mixtape} Chapter 6, \cite{hungtintonklein2023effect} Chapter 20
  \end{itemize}
  \item \textbf{Applications:} College wage premia, Peru's Mining \textit{Mita}, class sizes
  \begin{itemize}
    \item Readings: \textbf{\cite{dell2010mita}}, \textbf{\cite{zimmerman2014returns}}, \textbf{\cite{angrist1999maimonides}}, \cite{chetty2023diversifying}
  \end{itemize}
\end{itemize}
%\url{https://www.annualreviews.org/doi/abs/10.1146/annurev-economics-082222-074352?casa_token=5OzCf9NSgEMAAAAA%3ACwCox17aNu0Rrg5QnBOIzE8cXMfgDgYOgCgx2Ow7fP32tJj5H85AilUS2a4giTbbeip5Rkes9AgI}
%\url{https://www.nber.org/papers/w26380}
%Uber work by List
%AirBNB field experiments? Discrimination and audits...
%Slot functions here?
\week{Week 9} Machine Learning I
\begin{itemize}
  \item \textbf{Skills:} Decision Trees
  \begin{itemize}
    \item Readings: \textbf{\cite{athey2019machine}}, \textbf{\cite{varian2014bigdata}}, \cite{mullainathan2017machine}, \cite{kleinberg2015prediction}
  \end{itemize}
  \item \textbf{Application:} Bias and Judicial Decisions
  \begin{itemize}
    \item Readings: \textbf{\cite{kleinberg2018human}}, \cite{bertrand2004emily}, \cite{simonsohn2023bertrand}
  \end{itemize}
  \item \textit{\textbf{Problem Set 4 due 11/6 at Midnight}}
\end{itemize}
\week{Week 10} Machine Learning II
\begin{itemize}
  \item \textbf{Skills:} Regression penalization methods, Causal Forests
  \begin{itemize}
    \item Readings: \textbf{\cite{athey2019machine}}, \cite{varian2014bigdata}, \cite{mullainathan2017machine}, \cite{kleinberg2015prediction}
  \end{itemize}
  \item \textbf{Application:} Summer Jobs and At-Risk Youth
  \begin{itemize}
    \item Readings: \textbf{\cite{davis2017summer}}, \cite{naik2014streetscore}
  \end{itemize}
  \item \textit{\textbf{Data Description due 11/17 at Midnight}}
\end{itemize}
\week{Thanksgiving Recess} Gobble, gobble! 
\begin{itemize}
  \item \textit{\textbf{Problem Set 5 due 11/20 at Midnight}}
\end{itemize}
\week{Week 11} Text analysis I
\begin{itemize}
  \item \textbf{Skills:} Regular expressions, WordClouds, sentiment analysis
  \begin{itemize}
    \item Readings: \textbf{\cite{gentzkow2019text}}
  \end{itemize}
  \item \textbf{Application:} Google Flu Trends, Racial Animus and Elections
  \begin{itemize}
    \item Reading: \textbf{\cite{lazer2014parable}}, \textbf{\cite{stephensdavidowitz2014racial}}, \cite{ginsberg2009influenza}
  \end{itemize}
%\url{https://www.aeaweb.org/articles?id=10.1257/pandp.20231117}
%EJMR: https://www.aeaweb.org/articles?id=10.1257/pandp.20181101
%Bias in NLP: \url{https://www.brookings.edu/articles/detecting-and-mitigating-bias-in-natural-language-processing/}
\item \textit{\textbf{Problem Set 6 due 12/4 at Midnight}}
\end{itemize}
\week{Week 12} Text analysis II
\begin{itemize}
  \item \textbf{Skills:} Topics modeling, LLMs, AI
  \begin{itemize}
    \item Reading: \textbf{\cite{ash2023textalgorithms}}
  \end{itemize}
  \item \textbf{Application:} EJMR, Temperature and Twitter
  \begin{itemize}
    \item Readings: \textbf{\cite{wu2018gendered}}, \cite{moore2019temp}
  \end{itemize}
  \item \textit{\textbf{Problem Set 7 due 12/11 at Midnight}}
%EJMR: https://www.aeaweb.org/articles?id=10.1257/pandp.20181101
%Bias in NLP: \url{https://www.brookings.edu/articles/detecting-and-mitigating-bias-in-natural-language-processing/}
\end{itemize}

\textbf{Final Project due 12/11 at Midnight}

\subsection*{What is missing?}

\begin{itemize}
  \item Field and Quasiexperiments
  \item Data types, data storage
  \item Command line interface
  \item Optimization, vectorization
  \item Cluster computing
  \item Prediction and Machine Learning
  \item Cross-validation
  \item Supervised vs. unsupervised ML
  \item Bayesian ML
\end{itemize}

\newpage

%\bibliographystyle{AER}
\bibliographystyle{unsrtnat}
\bibliography{syllabus_bigdata}

%Field Experiments and Quasiexperiments %Maybe drop or add resume moving to opportunity?
% \begin{itemize}
%   \item \textbf{Skills:} Audit studies, correspondence studies
%   \begin{itemize}
%     \item Readings: \textbf{\cite{bertrand2004emily}}
%   \end{itemize}
%   \item \textbf{Application:} Resume studies and Racial Discrimination, Creating Moves to Opportunity
%   \begin{itemize}
%     \item Readings: \textbf{\cite{chetty2018neighborhoods}}, \textbf{\cite{chetty2019moving}}, \textbf{\cite{chetty2016moving}}
%   \end{itemize}
% \end{itemize}

\end{document}
